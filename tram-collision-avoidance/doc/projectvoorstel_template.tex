\documentclass[10pt]{article}
\usepackage[dutch]{babel}
\usepackage{a4}
\usepackage{amsmath}
\usepackage[pdftex]{graphicx}
\topmargin -0.5in \textheight 9.0in \oddsidemargin 0.0in
\evensidemargin 0.0in \textwidth 6.5in
\graphicspath{{images/}}

\begin{document}

\title{Template voor een projectvoorstel}
\author{ }
\date{ }
\maketitle

\section*{Structuur van het projectvoorstel}

\textbf{Er is een indicatief aantal pagina's vermeld waarin het redelijkerwijze moet mogelijk zijn om een goede projectbeschrijving weer te geven.} Eventuele randinformatie kan opgenomen worden in bijlagen.

\begin{tabbing}
\ref{sec:kerngegevens} \= Kerngegevens   (� 1/2 pag.) \\
\> \ref{sec:Titel}	Titel \\
\> \ref{sec:Contactpersoon}	Contactpersoon \\
\> \ref{sec:Uitvoerders}	Uitvoerders \\
\> \ref{sec:Startdatum}	Startdatum, doorlooptijd, menskracht \\
 \\
\ref{sec:Situering} Situering van het project   ( 1-2 pag.) \\
\> \ref{sec:Probleemstelling}	Probleemstelling \\
\> \ref{sec:Andere}	Vergelijkbare projecten van derden \\
 \\
\ref{sec:Beschrijving}  Beschrijving van het uit te voeren project (3-6 pag.) \\
\> \ref{sec:Doelstelling}	Doelstelling \\
\> \ref{sec:Aanpak}	Aanpak en werkprogramma \\
\> \ref{sec:Projectplanning}	Projectplanning \\
 \\
\ref{sec:Nut} Nut van de resultaten (� 1/2 pag.) \\
\> \ref{sec:BeschrijvingNut}	Beschrijving van het nut \\
\> \ref{sec:Overdracht}	Overdracht van kennis aan derden \\
 \\
\ref{sec:Gebruik} Gebruik van resultaten van derden (� 1 pag.) \\
 \\
\ref{sec:Bijlagen} Bijlagen (facultatief) \\
 \\
\end{tabbing}

\newpage

\section{Kerngegevens}\label{sec:kerngegevens}

\subsection{Titel}\label{sec:Titel}

Geef een korte zinvolle titel voor het project, die de essentie van het behandelde thema weergeeft. 


\subsection{Contactpersoon} \label{sec:Contactpersoon}

naam + email adres van de contactperso(o)n(en).


\subsection{Uitvoerders}\label{sec:Uitvoerders}

Naam en contactgegevens van de uitvoerders


\subsection{Startdatum, doorlooptijd, menskracht en begroting}\label{sec:Startdatum}

\begin{itemize}
\item	Startdatum voor het project

\item	Totale duur van het project (in weken)

\item	Totaal aantal mensuren dat zal gewerkt worden aan het project (zowel tijdens de contacturen als daarbuiten).

\end{itemize}

\section{Situering van het project}\label{sec:Situering}

\subsection{Probleemstelling}\label{sec:Probleemstelling}

Geef een korte beschrijving van het probleem: waarover gaat het, wat zijn de belangrijkste elementen, wat zijn de noden, welke technologie kan men (niet) gebruiken, waarom willen we een oplossing, welke andere aspecten mag men niet uit het oog verliezen, ...

Deze tekst dient leesbaar te zijn zowel voor een specialist in het domein ("heeft men een correcte situering gegeven van het probleem?"), als voor een leek ("waarom wil men hierover een innovatieproject verrichten?").




\subsection{Andere projecten van derden}\label{sec:Andere}
\textbf{\ldots die verband houden met dit voorstel \ldots }

Welke projecten zijn u bekend over hetzelfde probleem. Wat vond u terug op internet of in publicaties? Wat kan daarvan geleerd worden? Wat kan daarvan eventueel benut worden? Wat is de complementariteit of meerwaarde van onderstaand projectvoorstel t.o.v. de reeds bestaande initiatieven?


\section{Beschrijving van het uit te voeren project}\label{sec:Beschrijving}

\subsection{Doelstelling}\label{sec:Doelstelling}
De doelstelling geeft kort en krachtig weer wat u \textbf{wil bekomen met dit project}. Het is een \textbf{samenvatting} ($\pm$ \textbf{0,5 pg}) in doorlopende tekst, die de nadruk legt op problemen die zullen opgelost worden, de verhoopte resultaten van dit project en de meest realistische toepassingsmogelijkheden. Het legt niet de nadruk op de specifieke planning (het "hoe" of "wanneer"), noch op details of op de resultaten van eventuele andere aansluitende activiteiten. 


\subsection{Aanpak en werkprogramma}\label{sec:Aanpak}

\textbf{Hoofdstukken \ref{sec:Doelstelling}, \ref{sec:Aanpak} en \ref{sec:Projectplanning} vormen de kern van het hele voorstel. De kwaliteit van dit deel bepaalt meestal of een project wel of niet zal slagen. }\\
\\
Hier wordt aangegeven \textbf{\emph{wat} men \emph{concreet} gaat doen}, \emph{waarom} voor die bepaalde werkpakketten wordt geopteerd, en \emph{hoe} elke uitvoerder zijn werk zal aanpakken. Uit een degelijke en concrete takenbeschrijving zal de lezer zich een mening kunnen vormen over de resultaten die wel of niet zullen voortvloeien uit het project, hoe de taken zich verhouden t.o.v. wat al bestaat, hoeveel werk een bepaalde taak normalerwijze met zich meebrengt, of de voorgestelde activiteiten moeilijk dan wel gemakkelijk zijn, welke nieuwe resultaten men mag verwachten, welk nut we eraan zullen hebben, enz.

\begin{itemize}
\item	Geef eerst op \emph{algemene wijze} (1-2 pg) aan hoe u het gestelde probleem wenst aan te pakken en op te lossen. Geef eventueel een schematisch overzicht (met figuur of diagram) van de verschillende onderdelen en hun samenhang. Beperk het aantal werkpakketten tot een overzichtelijk geheel. 

\item	Bespreek daarna \emph{elk werkpakket op een adequate wijze}, o.a. met de voorziene hoeveelheid werk per aanvrager. Besteed liefst meer tekst en uitleg aan de belangrijkste en/of omvangrijkste werkpakketten, dan aan de kleinere of bijkomstige. Zet bij elke titel van een werkpakket onmiddellijk het aantal mensuren dat deze taak met zich meebrengt. 

\item	Duid bij een aantal werkpakketten ook de deliverables aan (software programma's, presentaties, documenten, software testen, \ldots). Wees daarbij zo concreet mogelijk (Wat is de input van het programma, wat is de output, hoe wordt het getest, wanneer is het klaar, is er feedback mogelijk door gebruikers?) Duid aan welke werkpakketten essentieel zijn voor het welslagen van het project. Welke werkpakketten liggen op het kritisch pad voor het eindresultaat. Welke werkpakketten zijn van secundair belang en kunnen tijdens de uitvoering weggesnoeid worden om tijd te winnen.

\end{itemize}

\subsection{Projectplanning}\label{sec:Projectplanning}

In de onderstaande tabel geeft u schematisch de timing en het aantal mensuren die men effectief op het project zal inzetten. Geef ook aan welke deadlines er zijn. Welke deliverable is wanneer klaar? Wanneer wordt er vergaderd? Wanneer wordt er gerapporteerd?

\begin{table}[h]
	\centering
		\begin{tabular}{|*{14}{c|}}
		\hline
		\textbf{Activiteit} & \multicolumn{12}{l|}{\textbf{Periode}} & \textbf{Opmerkingen} \\
		\hline
							 & \multicolumn{12}{l|}{\textbf{Semesterweken}} &   \\
		\hline 
		  & 1	& 2	& 3	& 4	& 5	& 6	& 7	& 8	& 9	& 10 & 11	& 12 & \\
		\hline 
		WP1: \ldots & & & & & & & & & & & & & \\
		\hline 
		WP2: \ldots & & & & & & & & & & & & & \\
		\hline 
		WP3: \ldots & & & & & & & & & & & & & \\
		\hline 
		 \ldots & & & & & & & & & & & & & \\
		\hline 
		 \ldots & & & & & & & & & & & & & \\
		\hline 
		Vakantie: \ldots & & & & & & & & & & & & & \\
		\hline
		\end{tabular}
	\caption{projectplanning}
	\label{tab:projectplanning}
\end{table}
																							

Zeg ook kort hoe de organisatie van het project zal verlopen. Wie co\"ordineert het project? Hoe verloopt de communicatie (via email, samenkomsten, \ldots) Wie rapporteert wanneer aan wie? Wordt er gebruik gemaakt van de Dokeos site?


\section{Nut van de resultaten}\label{sec:Nut}

\subsection{Beschrijving van het nut}\label{sec:BeschrijvingNut}

\emph{Hoe} zullen we de \emph{projectresultaten kunnen gebruiken}? Wat is bijvoorbeeld de meerwaarde voor studenten die in een volgende academiejaar dergelijke projecten zullen maken? Kunnen we de resultaten gebruiken op opendeurdagen? Kan het gebruikt worden in thesissen? Kan een programma enkel gebruikt worden door de auteurs van het programma of ook door leken, kinderen, scholieren?

Beschrijf  ook op een kwalitatieve en indien mogelijk op een kwantitatieve wijze de bredere doelgroep die ge\"interesseerd kan zijn in de resultaten van het project. Welke toegevoegde waarde kan het project hen bieden?  Indien zinvol kan u bepaalde categorie\"en gebruikers defini\"eren en voor elke categorie de specifieke voordelen beschrijven. Wees hierbij zo concreet mogelijk. 


\subsection{Overdracht van kennis aan de gebruikers: hoe?}\label{sec:Overdracht}

Een project heeft geen enkel nut als de resultaten niet kenbaar gemaakt worden. Bovendien moeten de resultaten onder een bruikbare vorm beschikbaar worden gesteld. Welke maatregelen zal u nemen om de kennis en de projectresultaten eventueel over te kunnen dragen aan de ge\"interesseerde gebruikers? (Presentaties, rapportering, demonstraties op opendeurdagen, internetsite, CD, \ldots) 

\section{Gebruik van resultaten van derden}\label{sec:Gebruik}

Vermeld hier de resultaten van derden waar u gebruik van zult maken (matlab toolboxes, matlab programma's op internet, demoprogramma's, testdata, powerpoint presentaties, tutorials, handboeken, boeken uit bibliotheek, pdf files\ldots) 

\section{Bijlagen} \label{sec:Bijlagen}
Bijkomend documentatiemateriaal (facultatief). 

\end{document}