% Standaardinstellingen
\documentclass[a4paper,oneside,11pt,final]{memoir}

% Noot: zorg ervoor dat Nederlandse woord-splitsing geactiveerd is.
\usepackage[dutch]{babel}
%\usepackage[none]{hyphenat}
% UTF8 gebruiken voor gebruik van alle symbolen
\usepackage[utf8]{inputenc}
\usepackage{eurosym}

% Noot: je kan het graphicxpakket een optie dvips of pdftex doorgeven
% in dat geval moet je ze ook aan iiiscriptie doorgeven, dus bijvoorbeeld
% \usepackage[dvips]{graphicx}
% \usepackage[dvips]{iiiscriptie}
\usepackage{graphicx}
\usepackage{IIIScriptie}
\usepackage{caption}
%\usepackage{ccaption}

% Tabellen eleganter maken
\usepackage{booktabs}

% Navigeerbaarheid van hyperlinks in PDF
\usepackage{hyperref}

% Extra functies
% \input{functies.tex}

%
% Titelpagina
%

% Invullen velden
\departement{Departement Toegepaste Ingenieurswetenschappen}
\deptadres{Schoonmeersstraat 52 - 9000 Gent}
\studiejaar{1e Master Informatica}
\soortrapport{Projectvoorstel Beeldverwerking}
\title{Tram Collision Advoidance}
\author{Tim BESARD\\Ruben SCHOLLAERT\\Dimitri ROOSE\\Sebastiaan LABIJN}

% Pagina maken
\begin{document}
\maketitle

\section{Kerngegevens}\label{sec:kerngegevens}

\subsection{Titel}\label{sec:Titel}

Tram Collision Avoidance.

\subsection{Contactpersoon} \label{sec:Contactpersoon}

\begin{table}[h]
		\begin{tabular}{*{14}{l}}
		Dimitri Roose & roose.dimitri@gmail.com \\
		\end{tabular}
\end{table}

\subsection{Uitvoerders}\label{sec:Uitvoerders}

\begin{table}[h]
		\begin{tabular}{*{14}{l}}
		Tim Besard & tim.besard@gmail.com \\
		Sebastiaan Labijn & sebastiaan.labijn@gmail.com \\
		Ruben Schollaert & ruben.schollaert@telenet.be \\	
		\end{tabular}
\end{table}

\subsection{Startdatum, doorlooptijd, menskracht en begroting}\label{sec:Startdatum}
%TODO
\begin{table}[h]
		\begin{tabular}{*{14}{l}}
		Startdatum & 17 februari 2011 \\
		Doorlooptijd & 103 dagen\\
		Menskracht & 160\\	
		Begroting & Niet van toepassing.\\
		\end{tabular}
\end{table}

\section*{Situering van het project}\label{sec:Situering}

\subsection{Probleemstelling}\label{sec:Probleemstelling}

%Geef een korte beschrijving van het probleem: waarover gaat het, wat zijn de belangrijkste elementen, wat zijn de noden, welke technologie kan men (niet) gebruiken, waarom willen we een oplossing, welke andere aspecten mag men niet uit het oog verliezen, ...
%
%Deze tekst dient leesbaar te zijn zowel voor een specialist in het domein ("heeft men een correcte situering gegeven van het probleem?"), als voor een leek ("waarom wil men hierover een innovatieproject verrichten?").

Er gebeuren jaarlijks heel wat ongelukken met trams omdat de remafstand vrij groot is en er geen uitwijkmogelijkheden zijn. In dit project proberen we dat te vermijden door via computervisietechnieken een schatting te maken van de afstand tot andere trams en tot andere voertuigen of weggebruikers die de sporen blokkeren. 

Niet enkel de afstand vanuit de tram is belangrijk, ook de positie van de pantograaf en de leidraad hebben hun invloed op de voortstuwing van de tram, en dan vooral de slijtage. Een onderdeel van dit project is het detecteren van de positie van pantograaf en leidraad in videosequenties opgenomen vanop het dak van een tram.

\subsection{Andere projecten van derden}\label{sec:Andere}
%TODO
%\textbf{\ldots die verband houden met dit voorstel \ldots }

%Welke projecten zijn u bekend over hetzelfde probleem. Wat vond u terug op internet of in publicaties? Wat kan daarvan geleerd worden? Wat kan daarvan eventueel benut worden? Wat is de complementariteit of meerwaarde van onderstaand projectvoorstel t.o.v. de reeds bestaande initiatieven?

\section*{Beschrijving van het uit te voeren project}\label{sec:Beschrijving}

\subsection{Doelstelling}\label{sec:Doelstelling}

%De doelstelling geeft kort en krachtig weer wat u \textbf{wil bekomen met dit project}. Het is een \textbf{samenvatting} ($\pm$ \textbf{0,5 pg}) in doorlopende tekst, die de nadruk legt op problemen die zullen opgelost worden, de verhoopte resultaten van dit project en de meest realistische toepassingsmogelijkheden. Het legt niet de nadruk op de specifieke planning (het "hoe" of "wanneer"), noch op details of op de resultaten van eventuele andere aansluitende activiteiten.

Het doel van dit project is om aan de hand van videobeelden in real-time te kunnen beslissen of de tram een noodactie moet ondernemen om een mogelijke botsing te vermijden. De beelden worden gehaald uit een digitale videocamera, die zicht heeft op wat er zich voor de tram in kwestie gebeurd.

De beslissing om een noodstop uit te voeren, kan leiden uit verschillende situaties. Zo kunnen de videobeelden indiceren dat er zich een stilstaande tram bevindt op hetzelfde spoor. Indien de berekende afstand tussen de twee tramtoestellen te klein wordt, kan de beslissing genomen worden om te vertragen of zelfs om helemaal te stoppen.

Het gebeurt echter ook geregeld dat een tram botst met vreemde objecten, zoals auto's, personen, of andere voorwerpen. Hiertoe berusten we opnieuw op de videobeelden, waarna we opnieuw tot de beslissing kunnen overgaan om de tram tot stilstand te laten komen om zo een (mogelijk gevaarlijke) collisie te vermijden.


\subsection{Aanpak en werkprogramma}\label{sec:Aanpak}

%\textbf{Hoofdstukken \ref{sec:Doelstelling}, \ref{sec:Aanpak} en \ref{sec:Projectplanning} vormen de kern van het hele voorstel. De kwaliteit van dit deel bepaalt meestal of een project wel of niet zal slagen. }\\
%\\
%Hier wordt aangegeven \textbf{\emph{wat} men \emph{concreet} gaat doen}, \emph{waarom} voor die bepaalde werkpakketten wordt geopteerd, en \emph{hoe} elke uitvoerder zijn werk zal aanpakken. Uit een degelijke en concrete takenbeschrijving zal de lezer zich een mening kunnen vormen over de resultaten die wel of niet zullen voortvloeien uit het project, hoe de taken zich verhouden t.o.v. wat al bestaat, hoeveel werk een bepaalde taak normalerwijze met zich meebrengt, of de voorgestelde activiteiten moeilijk dan wel gemakkelijk zijn, welke nieuwe resultaten men mag verwachten, welk nut we eraan zullen hebben, enz.


\subsubsection{Algemene werkwijze}

% Geef eerst op \emph{algemene wijze} (1-2 pg) aan hoe u het gestelde probleem wenst aan te pakken en op te lossen. Geef eventueel een schematisch overzicht (met figuur of diagram) van de verschillende onderdelen en hun samenhang. Beperk het aantal werkpakketten tot een overzichtelijk geheel. 

\subsubsection{Beschrijving werkpakketten}

% Bespreek daarna \emph{elk werkpakket op een adequate wijze}, o.a. met de voorziene hoeveelheid werk per aanvrager. Besteed liefst meer tekst en uitleg aan de belangrijkste en/of omvangrijkste werkpakketten, dan aan de kleinere of bijkomstige. Zet bij elke titel van een werkpakket onmiddellijk het aantal mensuren dat deze taak met zich meebrengt. 

% Duid bij een aantal werkpakketten ook de deliverables aan (software programma's, presentaties, documenten, software testen, \ldots). Wees daarbij zo concreet mogelijk (Wat is de input van het programma, wat is de output, hoe wordt het getest, wanneer is het klaar, is er feedback mogelijk door gebruikers?) Duid aan welke werkpakketten essentieel zijn voor het welslagen van het project. Welke werkpakketten liggen op het kritisch pad voor het eindresultaat. Welke werkpakketten zijn van secundair belang en kunnen tijdens de uitvoering weggesnoeid worden om tijd te winnen.

\paragraph{Werkpakket 1 - Tramspoor detectie (30 uur)}

Het eerste werkpakket bestaat eruit om de sporen te detecteren voor de tram. Hiertoe zullen we edge-detection algoritmes toepassen (zoals Sobelfilters), waarna we de effectieve lijnen kunnen extraheren (met vb. de Canny edge detector). De meest courante algoritmen detecteren echter enkel rechte lijnen, terwijl tramsporen best gebogen kunnen zijn (hetzij beperkt). Er zal moeten onderzocht worden of er rond die beperkte kromming kan gewerkt worden, bijvoorbeeld door meerdere kortere lijnen te verbinden, of dat er een geavanceerder algoritme zal benodigd zijn.

Dit werkpakket is van kritiek belang, daar verschillende van de andere werkpakketten berusten op de correcte werking ervan. Om de andere werkpakketen toch toe te laten reeds te starten zonder dat dit pakket afgewerkt is, zal duidelijk moeten afgesproken worden onder welke vorm het uiteindelijke resultaat zal doorgegeven worden. Dit kan bijvoorbeeld onder de vorm van een puntencollectie, of beter, door een enkel object dat een set Tramsporen voorstelt, en voor elk van de twee sporen daarbij een collectie lijnstukken bevat.

\paragraph{Werkpakket 2 - Tramdetectie (50 uur)}
De trams dienen herkend te worden in de opeenvolgende frames. Er zijn verschillende technieken mogelijk om dit te verwezenlijken. Om een overzicht te krijgen over de efficiëntie van de technieken worden er enkele geïmplementeerd. Deze worden dan uitvoerig getest en aan de hand van de testresultaten.
\\\\
Een eerste manier is gebruik maken van template matching technieken.
Template matching kan nog eens worden onderverdeeld in twee benaderingen: feature-based en template-gebaseerde matching. 

De feature-based benadering maakt gebruik van de kenmerken van het zoek-en sjabloonafbeelding, zoals randen of hoeken, om de best passende locatie van het sjabloon in de bronafbeelding te zoeken.
De template-based, of mondiale aanpak, maakt gebruik van het gehele sjabloon.
Aan de hand van somvergelijkingen van metrische waarden zoals cross-correlatie, vierkantswortels of 
verschillen in absolute waarden, ken men opnieuw de meest levensvatbare plaats voor het sjabloon in de bronafbeelding vinden.
 
Als het sjabloon sterke kenmerken heeft, kan een feature-based aanpak worden overwogen. Aangezien deze aanpak geen rekening houdt met het geheel van het sjabloon, kan dit leiden tot een hogere computationele efficiëntie.

Voor sjablonen zonder sterke kenmerken kan een template-gebaseerde benadering efficiënt zijn. Aangezien template-gebaseerde sjabloon matching mogelijks het bemonsteren van een groot aantal punten vereist kan men dit verminderen door de resolutie de zoek- en sjabloon afbeelding te verminderen met dezelfde factor en op deze afgeslankte afbeeldingen het algoritme uit te voeren. 
In OpenCV biedt de keuze uit zes verschillende berekeningsmethoden:
\begin{itemize}
\item Correlatie coëfficient.
\item Kwadratisch verschil.
\item Cross correlatie.
\end{itemize}
Deze kunnen al dan niet genormeerd zijn. Er wordt wel nog geen rekening gehouden met schaling en/of rotatie.
\\\\
De tweede manier is gebruik maken van HAAR-training. Hierbij wordt een classificator getraind door middel van voorbeelden van trams. Eens getraind kan deze classificator dan gebruikt worden om in nieuwe bronafbeeldingen te zoeken naar het sjabloon. Om HAAR-training toe te passen is echter een dataset met afbeeldingen van trams nodig en deze is niet beschikbaar.
\\\\
Er wordt eerst een vergelijking gemaakt tussen implementaties van enkele feature-gebaseerde en template-gebaseerde benaderingen.
\paragraph{Werkpakket 3 - Berekenen afstand tot voorgaande tram (40 uur)}

Om de afstand tot de vorige tram te kunnen berekenen hebben we nood aan vast gegeven. Hiervoor nemen we als referentie dus de afstand tussen 2 tramsporen. Omdat de beelden onder een bepaalde hoek geschoten zijn, moeten we ook rekening houden met het perspectief. We hebben dus nood aan een algoritme dat het perspectief wegwerkt.

Hiervoor is het nodig om een homografie op te stellen die een bepaald perspectief steeds in hetzelfde vlak transformeert. In openCV wordt voorzien in homografie door de functie cvFindHomography. De uiteindelijke afstandbepaling kan dan gebeuren op basis van wiskundige berekening met betrekking tot afstanden.

\paragraph{Werkpakket 4 - Blokkerend object detecteren (50 uur)}

In eerste instantie gebruiken we bij dit werkpunt de resultaten van de vorige werkpunten, zo kunnen we aannemen dat er een obstakel is van zodra de tramsporen niet meer continu gedetecteerd worden.
\\\\
Rest ons nu nog aan te geven welk soort obstakel dit is. Aangezien we onmogelijk voor alle soorten objecten een herkenning kunnen maken beperken we ons hier tot het detecteren van auto's en mensen, de overige obstakels worden dan aangegeven als onbekend.
\\\\
Om mensen te detecteren gaan we gebruik maken van de kenmerkingen (descriptoren) van een persoon, zo hebben we de typische omega-vorm van het hoofd en de  bouw van een lichaam van een persoon die rechtop loopt. De combinatie van een hoofd en schouders kunnen we samen opnemen in een verzameling van descriptoren, die we dan testen tegen enkele testdata (postieve \& negatieve data) om de resultaten te bekomen die we wensen. Enkel het hoofd kunnen we detecteren via eenvoudige technieken zoals bounding boxes, OpenCV voorziet hiervoor in functies om dit dan op te halen uit beelden.
\\\\
Om auto's de detecteren gaan we gebruik maken van Haar classifier training. Hiervoor zullen we aan de hand van een collectie voorbeelden van auto's in zijaanzicht de classifier trainen. Eenmaal deze classifier getraind is kan deze gebruikt worden om auto's op te sporen in de beelden. Ook voor HAAR-training heeft OpenCV een ingebouwd systeem om deze training door te voeren.

\subsection{Projectplanning}\label{sec:Projectplanning}

In de onderstaande tabel geeft u schematisch de timing en het aantal mensuren die men effectief op het project zal inzetten. Geef ook aan welke deadlines er zijn. Welke deliverable is wanneer klaar? Wanneer wordt er vergaderd? Wanneer wordt er gerapporteerd?

\begin{table}[h]
	\centering
		\begin{tabular}{|*{14}{c|}}
		\hline
		\textbf{Activiteit} & \multicolumn{12}{l|}{\textbf{Periode}} & \textbf{Opmerkingen} \\
		\hline
							 & \multicolumn{12}{l|}{\textbf{Semesterweken}} &   \\
		\hline 
		  & 1	& 2	& 3	& 4	& 5	& 6	& 7	& 8	& 9	& 10 & 11	& 12 & \\
		\hline 
		WP1: \ldots &  &  &  &  & & & & & & & & & \\
		\hline 
		WP2: OpenCV Template Matching &  3 & 3 & 2 & 8 & & & & & & & & & \\
		\hline 
		WP2: Implementatie Template Matching & & & & &10&10& & & & & & & \\
		\hline 
		WP2: Testen Template Matching & & & & & & & &10&10&& & &\\
		\hline 
		WP3: \ldots & & & & & & & & & & & & & \\
		\hline 
		WP4: Blokkerend object detecteren& & & & & & & & & & & & & \\
		\hline
		WP4: Interpreteren resultaten & & & & & & & & & & & & & \\
		vorige WP& & & & & & & & & & & & & \\
		\hline
		WP4: Detectie personen adhv & & & & & & & & & & & & & \\
		 descriptoren \& bounding boxes& & & & & & & & & & & & & \\
		\hline
		WP4: HAAR-training auto's & & & & & & & & & & & & & \\
		+ detectie & & & & & & & & & & & & & \\
		\hline 
		Vakantie: \ldots & & & & & & & & & & & & & \\
		\hline
		\end{tabular}
	\caption{projectplanning}
	\label{tab:projectplanning}
\end{table}
																							

% Zeg ook kort hoe de organisatie van het project zal verlopen. Wie co\"ordineert het project? Hoe verloopt de communicatie (via email, samenkomsten, \ldots) Wie rapporteert wanneer aan wie? Wordt er gebruik gemaakt van de Dokeos site?

Het verloop van het project zal hoofdzakelijk democratisch bepaald worden. Zo hebben alle medewerkers evenwaardige toegang tot de centrale code-repository. De gebruikte service hiervoor kent echter een veel bredere functionaliteitswaaier, die de teamleden zal toelaten op efficiënte wijze te communiceren en overleggen over het verloop van het project. Zo wordt bijvoorbeeld voorzien in een wikipedia-achtige sectie, dat kan gebruikt worden om bepaalde algoritmes uit te doeken te doen, of om snel informatie neer te pennen die dan in een later stadium in het verslag kan verwerk worden. Ook voorziet dezelfde service in mogelijkheden om commentaar in te sturen over bepaalde codefragmenten of -toevoegingen, zodat het eenvoudig is om in team constructief te discussiëren over het specifieke verloop van een bepaald projectgedeelte.


\section*{Nut van de resultaten}\label{sec:Nut}

\subsection{Beschrijving van het nut}\label{sec:BeschrijvingNut}

%\emph{Hoe} zullen we de \emph{projectresultaten kunnen gebruiken}? Wat is bijvoorbeeld de meerwaarde voor studenten die in een volgende academiejaar dergelijke projecten zullen maken? Kunnen we de resultaten gebruiken op opendeurdagen? Kan het gebruikt worden in thesissen? Kan een programma enkel gebruikt worden door de auteurs van het programma of ook door leken, kinderen, scholieren?

%Beschrijf  ook op een kwalitatieve en indien mogelijk op een kwantitatieve wijze de bredere doelgroep die geïnteresseerd kan zijn in de resultaten van het project. Welke toegevoegde waarde kan het project hen bieden?  Indien zinvol kan u bepaalde categorieën gebruikers definiëren en voor elke categorie de specifieke voordelen beschrijven. Wees hierbij zo concreet mogelijk. 

Het uitgewerkte project zal een applicatie zijn die kan gebruikt worden door \emph{De Lijn}, deze applicatie kan dan gekoppeld worden aan een alarm of automatisch stopsysteem als een collision gedetecteerd wordt. Aangezien dit project specifiek toegepast wordt op tramsporen en trams van \emph{De Lijn} kan het niet zomaar toegepast worden voor andere doeleinden, wel zal elk apart werkpunt kunnen gebruikt worden buiten het project.
Zo kan men werkpunt 4 gebruiken om ook bij andere voertuigen voetgangers of andere blokkerende objecten te detecteren.

\subsection{Overdracht van kennis aan de gebruikers}\label{sec:Overdracht}

% Een project heeft geen enkel nut als de resultaten niet kenbaar gemaakt worden. Bovendien moeten de resultaten onder een bruikbare vorm beschikbaar worden gesteld. Welke maatregelen zal u nemen om de kennis en de projectresultaten eventueel over te kunnen dragen aan de geïnteresseerde gebruikers? (Presentaties, rapportering, demonstraties op opendeurdagen, internetsite, CD, \ldots) 

Om de kennis opgedaan met dit project over te dragen aan andere gebruikers, zullen we voorzien in verschillende hulpmiddellen, zowel theoretisch als concreet.

Het theoretische aspect wordt verzorgd door een paper die de beslissingen en overwegingen nodig om tot het begoogde resultaat te bekomen, voldoende uit de doeken doet. Zo zal een gebruiker probleemloos kunnen nalezen waartoe een bepaald algoritme toegepast is, alsook waarom we het verkozen hebben boven andere alternatieven.

Het praktisch aspect wordt verwezenlijkt door de geschreven code in herbruikbare vorm publiek beschikbaar te maken. Zo zal duidelijk worden hoe de gekozen algoritmen exact zijn toegepast, en hoe ze gecombineerd zijn om een sluitend geheel te vormen. De code zal ook gebundeld zijn met voorbeelden zodat de gebruiker eenvoudig in staat is om de werking van het pakket te verifiëren.

Tenslotte wordt ook voorzien in een presentatie, die zowel het theoretische als het meer praktische van het project oppervlakkig aanraakt om een nieuwe gebruiker een algemeen beeld te geven over het doel en werking van het project.

\section*{Gebruik van resultaten van derden}\label{sec:Gebruik}

% Vermeld hier de resultaten van derden waar u gebruik van zult maken (matlab toolboxes, matlab programma's op internet, demoprogramma's, testdata, powerpoint presentaties, tutorials, handboeken, boeken uit bibliotheek, pdf files\ldots) 

\section*{Bijlagen} \label{sec:Bijlagen}
Bijkomend documentatiemateriaal (facultatief). 

\end{document}
