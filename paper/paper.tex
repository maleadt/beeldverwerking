% Template for ICIP-2009 paper; to be used with:
%          spconf.sty  - ICASSP/ICIP LaTeX style file, and
%          IEEEbib.bst - IEEE bibliography style file.
% --------------------------------------------------------------------------
\documentclass{article}
\usepackage{spconf,amsmath,epsfig}

% Example definitions.
% --------------------
\def\x{{\mathbf x}}
\def\L{{\cal L}}

% Title.
% ------
\title{PEDESTRIAN DETECTION USING HAAR-LIKE FEATURES}
%
% Single address.
% ---------------
\name{Tim Besard, Ruben Schollaert, Dimitri Roose, Sebastiaan Labijn
\thanks{Thanks to University College Ghent for funding.}}
\address{University College Ghent\\Applied Engineering\\Campus Schoonmeersen, Valentin Vaerwyckweg 1, 9000 Gent}


\begin{document}
%\ninept
%
\maketitle
%
\begin{abstract}
The abstract should appear at the top of the left-hand column of text, about
0.5 inch (12 mm) below the title area and no more than 3.125 inches (80 mm) in
length.  Leave a 0.5 inch (12 mm) space between the end of the abstract and the
beginning of the main text.  The abstract should contain about 100 to 150
words, and should be identical to the abstract text submitted electronically
along with the paper cover sheet.  All manuscripts must be in English, printed
in black ink.
\end{abstract}
%
\begin{keywords}
Pedestrian Detection, Haar-like Features, AdaBoost Classification
\end{keywords}
%
\section{Introduction}
\label{sec:intro}
In this paper a solution for pedestrian detection is described using Haar-like features. It is being developed to be used in a tram collision advoidance project. 
\\
The major problems lie in~\cite{monteiro2006vision}:
\begin{itemize}
\item The size, color and style of pedestrian clothing are very diverse.
\item Pedestrians are nong-rigid bodies. Their shape and size differ greatly and therefore they are more complex that rigid objects.
\item Not only the pedestrians themselves are part of the problem. The background can cause cluttering and due this parts of it could be mistaken for pedestrians.
\item Illumination and weather conditions can lower the distinction between background and pedestrian.
\item Also the lack in motion due the slow speed of pedestrians hardens the possibility for them to be detected.
\end{itemize}
\section{Formatting your paper}
\label{sec:format}

All printed material, including text, illustrations, and charts, must be kept
within a print area of 7 inches (178 mm) wide by 9 inches (229 mm) high. Do
not write or print anything outside the print area. The top margin must be 1
inch (25 mm), except for the title page, and the left margin must be 0.75 inch
(19 mm).  All {\it text} must be in a two-column format. Columns are to be 3.39
inches (86 mm) wide, with a 0.24 inch (6 mm) space between them. Text must be
fully justified.

\section{PAGE TITLE SECTION}
\label{sec:pagestyle}

The paper title (on the first page) should begin 1.38 inches (35 mm) from the
top edge of the page, centered, completely capitalized, and in Times 14-point,
boldface type.  The authors' name(s) and affiliation(s) appear below the title
in capital and lower case letters.  Papers with multiple authors and
affiliations may require two or more lines for this information.

\section{TYPE-STYLE AND FONTS}
\label{sec:typestyle}

To achieve the best rendering both in the proceedings and from the CD-ROM, we
strongly encourage you to use Times-Roman font.  In addition, this will give
the proceedings a more uniform look.  Use a font that is no smaller than nine
point type throughout the paper, including figure captions.

In nine point type font, capital letters are 2 mm high.  {\bf If you use the
smallest point size, there should be no more than 3.2 lines/cm (8 lines/inch)
vertically.}  This is a minimum spacing; 2.75 lines/cm (7 lines/inch) will make
the paper much more readable.  Larger type sizes require correspondingly larger
vertical spacing.  Please do not double-space your paper.  TrueType or
Postscript Type 1 fonts are preferred.

The first paragraph in each section should not be indented, but all the
following paragraphs within the section should be indented as these paragraphs
demonstrate.

\section{MAJOR HEADINGS}
\label{sec:majhead}

Major headings, for example, "1. Introduction", should appear in all capital
letters, bold face if possible, centered in the column, with one blank line
before, and one blank line after. Use a period (".") after the heading number,
not a colon.

\subsection{Subheadings}
\label{ssec:subhead}

Subheadings should appear in lower case (initial word capitalized) in
boldface.  They should start at the left margin on a separate line.
 
\subsubsection{Sub-subheadings}
\label{sssec:subsubhead}

Sub-subheadings, as in this paragraph, are discouraged. However, if you
must use them, they should appear in lower case (initial word
capitalized) and start at the left margin on a separate line, with paragraph
text beginning on the following line.  They should be in italics.

\section{PRINTING YOUR PAPER}
\label{sec:print}

Print your properly formatted text on high-quality, 8.5 x 11-inch white printer
paper. A4 paper is also acceptable, but please leave the extra 0.5 inch (12 mm)
empty at the BOTTOM of the page and follow the top and left margins as
specified.  If the last page of your paper is only partially filled, arrange
the columns so that they are evenly balanced if possible, rather than having
one long column.

In LaTeX, to start a new column (but not a new page) and help balance the
last-page column lengths, you can use the command ``$\backslash$pagebreak'' as
demonstrated on this page (see the LaTeX source below).

\section{PAGE NUMBERING}
\label{sec:page}

Please do {\bf not} paginate your paper.  Page numbers, session numbers, and
conference identification will be inserted when the paper is included in the
proceedings.

\section{ILLUSTRATIONS, GRAPHS, AND PHOTOGRAPHS}
\label{sec:illust}

Illustrations must appear within the designated margins.  They may span the two
columns.  If possible, position illustrations at the top of columns, rather
than in the middle or at the bottom.  Caption and number every illustration.
All halftone illustrations must be clear black and white prints.  Colors may be
used, but they should be selected so as to be readable when printed on a
black-only printer.

Since there are many ways, often incompatible, of including images (e.g., with
experimental results) in a LaTeX document, below is an example of how to do
this \cite{Lamp86}.

% Below is an example of how to insert images. Delete the ``\vspace'' line,
% uncomment the preceding line ``\centerline...'' and replace ``imageX.ps''
% with a suitable PostScript file name.
% -------------------------------------------------------------------------
\begin{figure}[htb]

\begin{minipage}[b]{1.0\linewidth}
  \centering
% \centerline{\epsfig{figure=image1.ps,width=8.5cm}}
  \vspace{2.0cm}
  \centerline{(a) Result 1}\medskip
\end{minipage}
%
\begin{minipage}[b]{.48\linewidth}
  \centering
% \centerline{\epsfig{figure=image3.ps,width=4.0cm}}
  \vspace{1.5cm}
  \centerline{(b) Results 3}\medskip
\end{minipage}
\hfill
\begin{minipage}[b]{0.48\linewidth}
  \centering
% \centerline{\epsfig{figure=image4.ps,width=4.0cm}}
  \vspace{1.5cm}
  \centerline{(c) Result 4}\medskip
\end{minipage}
%
\caption{Example of placing a figure with experimental results.}
\label{fig:res}
%
\end{figure}

% To start a new column (but not a new page) and help balance the last-page
% column length use \vfill\pagebreak.
% -------------------------------------------------------------------------
\vfill
\pagebreak


\section{FOOTNOTES}
\label{sec:foot}

Use footnotes sparingly (or not at all!) and place them at the bottom of the
column on the page on which they are referenced. Use Times 9-point type,
single-spaced. To help your readers, avoid using footnotes altogether and
include necessary peripheral observations in the text (within parentheses, if
you prefer, as in this sentence).


\section{COPYRIGHT FORMS}
\label{sec:copyright}

You must include your fully completed, signed IEEE copyright release form when
you submit your paper. We {\bf must} have this form before your paper can be
published in the proceedings.  The copyright form is available as a Word file,
a PDF file, and an HTML file. You can also use the form sent with your author
kit.

\section{REFERENCES}
\label{sec:ref}

List and number all bibliographical references at the end of the paper.  The references can be numbered in alphabetic order or in order of appearance in the document.  When referring to them in the text, type the corresponding reference number in square brackets as shown at the end of this sentence \cite{C2}.
This is another reference \cite{GW}. More info concerning the use of BibTeX on wikipedia \cite{wiki}.

% Gebruik van bibliografie�n en BibTeX:
% Bibliografische verwijzingen voeg je toe in het bestand 'bibliografie.bib'. 
% In dat bestand vind je voorbeelden hoe je die moet toevoegen. 
% De meest 'cleane' manier is uiteraard een teksteditor gebruiken, 
% hoewel er ook programma's zijn die je een grafische interface geven om eenvoudig entry's toe te voegen. 
% Op OSX raad ik BibDesk aan.

% Hoe compileren met BibTeX?
% 1. Onderaan het document voeg je de bibliotheek met entry's toe (zie hieronder)
% 2. Compileer je LaTeX file (dit file dus) met LaTeX.
% 3. Indien dat geen errors van LaTeX oplevert, compileer je dat LaTeX file (dit file dus) met BibTeX.
% 4. Indien dat geen errors van BibTeX oplevert, compileer je twee (2) maal met LaTeX.
% 5. Done!

% References should be produced using the bibtex program from suitable
% BiBTeX files (here: bibliografie). The IEEEbib.bst bibliography
% style file from IEEE produces unsorted bibliography list.
% -------------------------------------------------------------------------
\bibliographystyle{IEEEbib}
\bibliography{bibliografie}

\end{document}
